\anonsection{Введение}

Облачные услуги --- это способ предоставления, потребления и управления технологией.
Они выводят гибкость и эффективность на революционный уровень, путем эволюции способов управления, таких как резервирование, самообслуживание, безопасность и непрерывность, которые соединяют физическую и виртуальную среду.
Следовательно, возрастает потребность в качественно продуманной архитектуре, позволяющей правильно и надежно организовать облачную инфраструктуру.

Для эффективной облачной инфраструктуры требуется эффективная структура и организация.
Определение и использование стандартов на каждом этапе работы, при размещении в стойках от отдельных компьютеров до отдельных кабелей и от рядовых операций до безопасности, позволяет сэкономить значительное время и правильно организовать процессы.

Чтобы спланировать и выполнить план, не нужны гигантские усилия.
Небольшая команда из специалистов и бизнес-пользователей может создать обоснованный план и организовать свою работу в облачной инфраструктуре.
Эта выделенная группа может намного эффективнее построить и управлять нестандартной облачной инфраструктурой, чем если компании будут просто продолжать добавлять дополнительные серверы и сервисы для поддержки центра обработки данных.

IaaS (Infrastructure as a Service) --- это предоставление пользователю компьютерной и сетевой инфраструктуры и их обслуживание как услуги в форме виртуализации, то есть виртуальной инфраструктуры.
Другими словами, на базе физической инфраструктуры дата-центров (ДЦ) провайдер создает виртуальную инфраструктуру, которую предоставляет пользователям как сервис.

Технология виртуализации ресурсов позволяет физическое оборудование (серверы, хранилища данных, сети передачи данных) разделить между пользователями на несколько частей, которые используются ими для выполнения текущих задач.
Например, на одном физическом сервере можно запустить сотни виртуальных серверов, а пользователю для решения задач выделить время доступа к ним.

\clearpage
