\section{Руководство администратора}

\subsection{Расположение серверов в сетевой инфраструктуре}
Во всей инфраструктуре на физических серверах первого дата-центра располагаются:
\begin{itemize}
  \item shared-хостинг (10.0.0.100);
  \item виртуализация OpenVZ (10.0.0.200);
  \item виртуализация KVM (10.0.0.300);
  \item сервер резервного копирования (10.0.0.400);
  \item физические сервера клиентов (10.0.0.500 --- 10.0.0.700).
\end{itemize}

В свою очередь некоторые важные сервисы инфраструктуры располагаются на виртуальных машинах, такие как:
\begin{itemize}
  \item сервер биллинга (10.0.0.201), OpenVZ;
  \item сервер управления IP-адесами (10.0.0.301), KVM;
  \item сервер DNS 1 (10.0.0.302), KVM.
\end{itemize}

Сервер мониторинга и сервер DNS 2 располагаются в виртуальных машинах KVM в другом дата-центре.
Сервер shared-хостинга выполняет роль главного (master) DNS-сервера.

\subsection{Сервер shared-хостинга}

Сервер shared-хостинга является самым уязвимым местом всей инфраструктуры из-за большого количества и плотности клиентов на сервере.
Распределение ресурсов происходит за счет встроенных в ядро Linux механизмов.

На сервере установлен и настроен веб-сервер в составе следующего ПО:
\begin{itemize}
  \item роль http-сервера выполняет связка между Apache HTTP Server, кэширующего и проксирующего http-сервера nginx и обработчика динамических запросов php-fpm;
  \item memcached кэширует динамические запросы в оперативной памяти;
  \item роли серверов баз данных выполняют MySQL и PostgreSQL.
\end{itemize}

nginx принимает запросы к серверу с порта 80 и решает каким образом обрабатывать запрос, если это запрос на получение статического файла, то nginx сам его обрабатывает и в случае необходимости кэширует.
Если же запрашивается динамическое содержимое, то nginx передает запрос на бэкенд, где его, в зависимости от настройки виртуального хоста сайта обрабатывает либо php-fpm, либо встроенный модуль Apache mod\_php.
nginx не может кэшировать динамические запросы, поэтому для этого имеется memcached, который работает в связке с Apache и позволяет кэшировать динамические запросы.
Для правильной работы Apache с memcached необходима установка модуля PHP php5-memcached:
\begin{lstlisting}
# apt-get install php5-memcached
\end{lstlisting}

В качестве связки для работы почты выступает SMTP-сервер Exim Internet Mailer, запросы по протоколам POP3 и IMAP принимает Dovecot.
В качестве антиспам-связки выступает Postgrey, Spamassasin, ClamAV и OpenDKIM.

ProFTPd необходим для работы пользователей по протоколу FTP.

Сервер управления временем NTP обращается к следующим серверам за уточнением времени на сервере:
\begin{lstlisting}
# cat /etc/ntp.conf | grep ^server | awk '{print $2}'
0.debian.pool.ntp.org
1.debian.pool.ntp.org
2.debian.pool.ntp.org
3.debian.pool.ntp.org
\end{lstlisting}

Конфигурация SSH требует отдельного пояснения.
Доступ к SSH по стандартному порту закрыт, это позволяет избавиться от большинства злоумышленников, которые пытаются скомпрометировать пароль доступа к серверу.
Также запрещен вход от пользователя root, для этого создан отдельный пользователь.
Также можно ограничить доступ к серверу по SSH, оставив только доверенные адреса или подсети адресов.
\begin{lstlisting}
Adding user for SSH:
# useradd sshuser
# passwd sshuser
SSH config:
# cat /etc/ssh/sshd_config
#change SSH port 222:
Port 222
#close root login:
PermitRootLogin yes
#allow IP-addresses:
Match host 10.0.0.111, 10.0.0.222
  #root for allowed IP's:
  PermitRootLogin yes
Restart SSH:
# service ssh restart
\end{lstlisting}

Разрешен вход на сервер по ключам:
\begin{lstlisting}
Key Generating:
user@10.0.0.111~$ ssh-keygen
user@10.0.0.111~$ ssh-keygen -p
Copy private key to server:
user@10.0.0.111~$ ssh-copy-id -p 222 root@10.0.0.100
Reconnect to server:
user@10.0.0.111~$ ssh -p 222 root@10.0.0.100
\end{lstlisting}

Резервные копии с сервера делаются средствами панели ISPmanager в ночное время, применяется инкрементальный метод резервного копирования, когда сначала делается полная копия, а затем инкрементальные копии только измененных данных.

На сервере функционирует скрипт блокировки адресов, которые слишком часто подбирают пароли доступа к панели администратора наиболее популярных CMS, таких как WordPress и Joomla.
Скрипт находится в открытом доступе и имеет простое использование:
\begin{lstlisting}
# cat /etc/nginx/blockips.sh
#!/bin/bash
# admin@amet13.name
# v0.2 30.06.2014
command=$(cat /var/log/nginx/access.log | \
grep "administrator/index.php \|wp-login.php " | \
cut -f1 -d " " | sort | \
uniq -c | sort -n | \
awk '{ if ($1 > 2999) print $2}')
# Add new ip's to database
for word in $command; do
   # Is that ip already in database?
   grep $word /etc/nginx/blockips.conf > /dev/null
   if (( $? )); then 
      sed -i "s/allow all;/deny $word;\nallow all;/" \
      /etc/nginx/blockips.conf
   fi
done
exit 0
\end{lstlisting}

Основные конфигурации настроек веб-сервера хранятся в:
\begin{lstlisting}
/etc/nginx/nginx.conf
/etc/apache2/apache2.conf
/etc/php5/{apache2,cgi,cli,fpm}/php.ini
/etc/mysql/my.cnf
/etc/postgresql/9.1/main/postgresql.conf
\end{lstlisting}

Для сканирования уязвимостей на сайтах клиентов используется утилита maldet, которая находит подозрительные файлы в системе и составляет отчет по найденным файлам.
Принцип пользования утилитой:
\begin{lstlisting}
Instalaltion:
# cd /tmp/
# wget http://www.rfxn.com/downloads/maldetect-current.tar.gz
# tar xfz maldetect-current.tar.gz
# cd maldetect-*
# ./install.sh
Run scanning (-b -- background mode):
# maldet -b --scan-all /var/www/
Show reports list:
# maldet --report list
Report view:
# maldet --report 021715-1414.3266
Send infected files to quarantine:
# maldet -q 021715-1414.3266
\end{lstlisting}

На сервере shared-хостинга в сайтах пользователей большое количество уязвимостей, которые часто используются для рассылки спама с сервера.
Для обнаружения большого числа рассылок с сервера написан плагин для системы мониторинга Nagios, который контролирует число писем в почтовой очереди, а также проверяет наиболее популярные организации, которые собирают данные о спам-серверах.

Для просмотра списка очереди используется команда exim -bp или mailq.
Таким образом можно просмотреть количество писем для разных доменов и их количество:
\begin{lstlisting}
# exim -bp | exiqsumm -c -s | head -10
Count  Volume  Oldest  Newest  Domain
-----  ------  ------  ------  ------
12151  1378KB     14h      0m  spamsite.ru > host.ru
   93   264KB     14h      5m  spamsite.ru > gmail.com
   16   154KB     13h     11h  site.ru > kk.com
    8    77KB     13h     11h  site.ru > ww.com
...
\end{lstlisting}

В данном случае очевидна рассылка писем с сайта spamsite.ru.
Посмотреть ID письма а также содержимое заголовков и тела письма можно командами:
\begin{lstlisting}
# exiqgrep -b -f 'spamsite.ru' | head -1
1YyRgs-0000zD-Qy From: <fobos@spamsite.ru> To: maria@host.ru
# exim -Mvh 1YyRgs-0000zD-Qy
# exim -Mvb 1YyRgs-0000zD-Qy
\end{lstlisting}

Если сайт действительно заражен, то следует отключить его, оповестить пользователя о закрытии сайта и необходимости избавиться от вредоносного кода и удалить очередь сообщений из этих писем:
\begin{lstlisting}
# exiqgrep -b -f 'spamsite.ru' | awk '{print $1}' | xargs exim -Mrm
\end{lstlisting}

В случае неработоспособности панели ISPmanager 5 возможна ее перезагрузка:
\begin{lstlisting}
# pgrep -l core
4437 core
7129 core
21194 core
# killall core
\end{lstlisting}

Обновление компонентов системы:
\begin{lstlisting}
# apt-get update
# apt-get upgrade
\end{lstlisting}

\subsection{Сервер виртуализации OpenVZ}



\clearpage
