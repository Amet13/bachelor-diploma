\section{Описание виртуальной инфраструктуры}

\subsection{Общая схема инфраструктуры}

Общая схема инфраструктуры представлена на рисунке \ref{infrast-scheme}.
\addimghere{infrast-scheme}{1}{Общая схема инфраструктуры}{infrast-scheme}

В дата-центре 1 располагается основная часть инфраструктуры: сервера shared-хостинга, виртуализации OpenVZ и KVM, сервер резервного копирования и выделенные сервера клиентов.

В дата-центре 2 на двух арендованных виртуальных машинах находится один из подчиненных DNS-серверов, а также сервер мониторинга.

На физических серверах располагаются важные элементы инфраструктуры: главный и один подчиненный DNS-сервера, система биллинга и система управления IP-адресами.

\subsection{Схема мониторинга}

Схема мониторинга инфраструктуры представлена на рисунке \ref{nagios-munin-scheme}.
\addimghere{nagios-munin-scheme}{1}{Схема мониторинга инфраструктуры}{nagios-munin-scheme}

В случае возникновения проблемы на одном из серверов, Nagios фиксирует изменения и отправляет текстовые уведомления по SMS и электронной почте дежурному администратору, который следит за системой мониторинга.
Nagios работает по протоколу NRPE (Nagios Remote Plugin Executor) и слушает порт 5666.

Munin имеет возможность визуализировать состояние ресурсов с помощью графиков, наблюдая за графиками можно делать выводы о событиях, происходящих в инфраструктуре.
Munin работает по протоколу MGF (Munin Graphing Framework) на порту 4949.

Помимо физических серверов, мониторятся также виртуальные машины, на которых располагаются сервисы инфраструктуры, а также некоторые виртуальные машины клиентов, по требованию.

\subsection{Схема репликации и балансировки нагрузки DNS-серверов}

С целью обеспечения отказоустойчивости системы была разработана следующая схема репликации и балансировки нагрузки DNS-серверов.
Схема представлена на рисунке \ref{dns-scheme}.
\addimghere{dns-scheme}{1}{Схема репликации и балансировки нагрузки DNS-серверов}{dns-scheme}

В каждой зоне есть один главный сервер имен на котором хранится официальная копия данных зоны.
Администратор модифицирует информацию, касающуюся зоны, редактируя файлы главного сервера \cite{unix-handbook}.
Подчиненный (slave) сервер копирует свои данные с главного сервера посредством операции, называемой передачей зоны (zone transfer).
В зоне имеется два подчиненных сервера, один из которых располагается в другом дата-центре.

На главном сервере установлен DNS-сервер PowerDNS.
На подчиненных серверах --- BIND 9.

Балансировка нагрузки осуществляется по алгоритму Round Robin (алгоритм кругового обслуживания).
Алгоритм представляет собой перебор по круговому циклу: первый запрос передается одному серверу, затем следующий запрос передается другому и так до достижения последнего сервера \cite{selectel}.

\subsection{Используемые технологии виртуализации}

На основе обзора литературы, проведенного в пункте \ref{literature}, выбраны и внедрены технологии виртуализации OpenVZ и KVM.

\subsection{Алгоритм функционирования инфраструктуры}

\clearpage
